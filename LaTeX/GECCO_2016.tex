\documentclass{sig-alternate}

\usepackage{times}
\usepackage{url}
\usepackage{todonotes}
\sloppy

\setlength{\parindent}{0.5cm} 

\newcommand{\citep}[1]{\cite{#1}}

% Copyright
\setcopyright{acmcopyright}
%\setcopyright{acmlicensed}
%\setcopyright{rightsretained}
%\setcopyright{usgov}
%\setcopyright{usgovmixed}
%\setcopyright{cagov}
%\setcopyright{cagovmixed}


% DOI
\doi{10.475/123_4}

% ISBN
\isbn{123-4567-24-567/08/06}

\acmPrice{\$15.00}

% You need the command \numberofauthors to handle the 'placement
% and alignment' of the authors beneath the title.
%
% For aesthetic reasons, we recommend 'three authors at a time'
% i.e. three 'name/affiliation blocks' be placed beneath the title.
%
% NOTE: You are NOT restricted in how many 'rows' of
% "name/affiliations" may appear. We just ask that you restrict
% the number of 'columns' to three.
%
% Because of the available 'opening page real-estate'
% we ask you to refrain from putting more than six authors
% (two rows with three columns) beneath the article title.
% More than six makes the first-page appear very cluttered indeed.
%
% Use the \alignauthor commands to handle the names
% and affiliations for an 'aesthetic maximum' of six authors.
% Add names, affiliations, addresses for
% the seventh etc. author(s) as the argument for the
% \additionalauthors command.
% These 'additional authors' will be output/set for you
% without further effort on your part as the last section in
% the body of your article BEFORE References or any Appendices.

\begin{document}

\conferenceinfo{GECCO'16,} {July 20-24, 2016, Denver, Colorado, USA.}
\CopyrightYear{2016}
\crdata{TBA}
\clubpenalty=10000
\widowpenalty = 10000

\title{Visualizing genetic programming ancestries using graph databases}

%\numberofauthors{4}
%\author{
%\alignauthor
%Nicholas Freitag McPhee\\
%	\affaddr{Div. of Science and Math}\\
%	\affaddr{Univ. of Minnesota, Morris}\\
%	\affaddr{Morris, MN USA-56267}\\
%	\email{mcphee@morris.umn.edu}
%\alignauthor
%Maggie M. Casale\\
%	\affaddr{Div. of Science and Math}\\
%	\affaddr{Univ. of Minnesota, Morris}\\
%	\affaddr{Morris, MN USA-56267}\\
%	\email{casal033@morris.umn.edu}
%\and
%\alignauthor
%Mitchell Finzel \\
%	\affaddr{Div. of Science and Math}\\
%	\affaddr{Univ. of Minnesota, Morris}\\
%	\affaddr{Morris, MN USA-56267}\\
%	\email{finze008@morris.umn.edu}
%\and
%\alignauthor
%Thomas Helmuth\\
%	\affaddr{Computer Science Dep't}\\
%	\affaddr{Washington and Lee Univ.}\\
%	\affaddr{Lexington, VA USA-24450}\\
%	\email{helmutht@wlu.edu}
%\alignauthor
%Lee Spector\\
%	\affaddr{Cognitive Science}\\
%	\affaddr{Hampshire College}\\
%	\affaddr{Amherst, MA USA-01002}\\
%	\email{lspector@hampshire.edu}
%}

\maketitle

\begin{abstract}


\end{abstract}

\begin{CCSXML}
	<ccs2012>
	<concept>
	<concept_id>10003120.10003145.10003146.10010892</concept_id>
	<concept_desc>Human-centered computing~Graph drawings</concept_desc>
	<concept_significance>500</concept_significance>
	</concept>
	<concept>
	<concept_id>10003120.10003145.10003147.10010364</concept_id>
	<concept_desc>Human-centered computing~Scientific visualization</concept_desc>
	<concept_significance>500</concept_significance>
	</concept>
	<concept>
	<concept_id>10010147.10010178.10010205.10010206</concept_id>
	<concept_desc>Computing methodologies~Heuristic function construction</concept_desc>
	<concept_significance>300</concept_significance>
	</concept>
	<concept>
	<concept_id>10010147.10010257.10010293.10011809.10011813</concept_id>
	<concept_desc>Computing methodologies~Genetic programming</concept_desc>
	<concept_significance>300</concept_significance>
	</concept>
	</ccs2012>
\end{CCSXML}

\ccsdesc[500]{Human-centered computing~Graph drawings}
\ccsdesc[500]{Human-centered computing~Scientific visualization}
\ccsdesc[300]{Computing methodologies~Heuristic function construction}
\ccsdesc[300]{Computing methodologies~Genetic programming}

\printccsdesc

\keywords{visualization; genetic programming; graph database; ancestry}

\section{Introduction}
\label{sec:introduction}

Reporting results of genetic programming (GP) and evolutionary computation is frequently limited to aggregate statistics such as mean best fitness or percentage of successful runs. Unfortunately this fails to convey the complex dynamics of such evolutionary systems and obscures or omits potentially valuable information about \emph{why} the runs behaved as they did. Previous work~\cite{McPhee:2015:GPTP} has demonstrated the utility of graph databases as tools for collecting and analyzing ancestry in GP runs, but was focused on sections of individual runs.

In this poster we illustrate the use of these tools as a means of exploring entire ancestry trees. We use the Titan graph database\footnote{\url{http://thinkaurelius.github.io/titan/}} along with the Gremlin shell and the Tinkerpop query tools\footnote{\url{https://tinkerpop.incubator.apache.org/}} to store the parent-child relationships from genetic programming runs, and to extract the ancestry trees of specified individuals. We then visualize these subgraphs using the Graphviz \texttt{dot} graph layout tool\footnote{\url{http://www.graphviz.org/}}.

% In this section we'll look at a successful run in RBM coloring. This run will 
% be shown in its full ancestry and a filtered version with the same colors of 
% the full run. This way we can compare these indviduals in the filtered version
% to that of the full one.
\section{Example: Comparing a Successful Run in Filtered \& Full}

% In the section we'll look at least one successful and failed run in dual
% coloring. This will allow us to see how individuals are prgressing (or not)
% over time. Having these ancestries in full will allow us to generally comapre
% the generation sizes and get the visual impact of a failed run.
\section{Example: Success vs. failure}
\label{sec:SuccessVsFail}


% In this section we'll look at multiple successful runs in their filtered
% versions. We'll create a RBM color scheme which incorporates all error vectors
% in the filtered ancestries. This will allow us to see similar individual in 
% other runs. Ex) You may see red at the top of one run, and at the end of another. Which could imply that oine found a solution earlier.
\section{Example: Comparing Filtered Successful Runs}



\bibliographystyle{abbrv}
\bibliography{GECCO_2016}

\end{document}